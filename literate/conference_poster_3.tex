%%%%%%%%%%%%%%%%%%%%%%%%%%%%%%%%%%%%%%%%%
% baposter Portrait Poster
% LaTeX Template
% Version 1.0 (15/5/13)
%
% Created by:
% Brian Amberg (baposter@brian-amberg.de)
%
% This template has been downloaded from:
% http://www.LaTeXTemplates.com
%
% License:
% CC BY-NC-SA 3.0 (http://creativecommons.org/licenses/by-nc-sa/3.0/)
%
%%%%%%%%%%%%%%%%%%%%%%%%%%%%%%%%%%%%%%%%%

%----------------------------------------------------------------------------------------
%	PACKAGES AND OTHER DOCUMENT CONFIGURATIONS
%----------------------------------------------------------------------------------------

%\documentclass[a0paper,portrait]{baposter}

\documentclass[paperwidth=42in,paperheight=44in,fontscale=0.31]{baposter}
\usepackage[font=small,labelfont=bf]{caption} % Required for specifying captions to tables and figures
\usepackage{booktabs} % Horizontal rules in tables
\usepackage{relsize} % Used for making text smaller in some places

\graphicspath{{figures/}} % Directory in which figures are stored

\definecolor{bordercol}{RGB}{40,40,40} % Border color of content boxes
\definecolor{headercol1}{RGB}{158,202,225} % Background color for the header in the content boxes (left side)
\definecolor{headercol2}{RGB}{49,130,189} % Background color for the header in the content boxes (right side)
\definecolor{headerfontcol}{RGB}{0,0,0} % Text color for the header text in the content boxes
\definecolor{boxcolor}{RGB}{222,235,247} % Background color for the content in the content boxes

\begin{document}

\background{ % Set the background to an image (background.pdf)
\begin{tikzpicture}[remember picture,overlay]
\draw (current page.north west)+(-2em,2em) node[anchor=north west]
{\includegraphics[height=1.1\textheight]{background}};
\end{tikzpicture}
}

\begin{poster}{
grid=false,
borderColor=bordercol, % Border color of content boxes
headerColorOne=headercol1, % Background color for the header in the content boxes (left side)
headerColorTwo=headercol2, % Background color for the header in the content boxes (right side)
headerFontColor=headerfontcol, % Text color for the header text in the content boxes
boxColorOne=boxcolor, % Background color for the content in the content boxes
headershape=roundedright, % Specify the rounded corner in the content box headers
headerfont=\Large\sf\bf, % Font modifiers for the text in the content box headers
textborder=rectangle,
%background=user,
bgColorOne=white, % Background color for the gradient on the left side of the poster
bgColorTwo=white, % Background color for the gradient on the right side of the poster
headerborder=open, % Change to closed for a line under the content box headers
boxshade=plain
}
{}
%
%----------------------------------------------------------------------------------------
%	TITLE AND AUTHOR NAME
%----------------------------------------------------------------------------------------
%
{\sf\bf Temporal and Spatial Trends in Winter Wheat Yields  \\ Monitored Via Multiple Online Data Sources } % Poster title
{\vspace{1em} Peter Claussen\\ % Author names
{\smaller Pete@gdmdata.com}} % Author email addresses
{\includegraphics[scale=0.40]{barcodes}} % University/lab logo

%----------------------------------------------------------------------------------------
%	INTRODUCTION
%----------------------------------------------------------------------------------------

\headerbox{Introduction}{name=introduction,column=0,row=0}{

Recent studies have suggested that crop yields gains have slowed or stagnated in many regions globally \cite{lin.m-07-2012,graybosch.r-2014}, while others shows patterns of increasing yield \cite{ray.d-12-2012} . A simple analysis of winter wheat yield in the central United States shows that while yield gains have slowed for some (southern) regions, relative to mid-century gains, yield improvement have accelerated in other (northern) regions (Figure \ref{prelim}a). This implies a shift in winter wheat productions zones, perhaps related to climate change.


To explore this shift, data from several public databases were combined for covariate analysis to determine if changes in environmental and socio-economic variables correlate with geospatial yield trends. For simplicity, this analysis was centered on the Great Plains states plus adjacent states with testing stations cooperating in the Hard Winter Wheat Regional Nursery \cite{hww-rpn}. (Figure \ref{prelim}b). The period from 1984-2014 was chosen by inspection of the inflection points in (Figure \ref{prelim}a).


\begin{center}
%   \includegraphics[width=0.49\linewidth]{WheatYields-centralyear}
%   \includegraphics[width=0.49\linewidth]{WheatYields-distance}
\includegraphics[width=0.99\linewidth]{WheatYields-intro}
   \captionof{figure}{{\smaller a. County average wheat yields (bu/acre) for six states in the Great Plains. b. Approximate distance (degrees latitude/longitude) from county center to nearest HWW RPN site.}}
   \label{prelim}
\end{center}

}

%----------------------------------------------------------------------------------------
%	MATERIALS AND METHODS
%----------------------------------------------------------------------------------------

\headerbox{Materials and Methods}{name=methods,column=0,below=introduction}{
 
 \smaller
Data sources and brief descriptions:
\begin{itemize}
   \item{USDA NASS} \cite{usda-nass}
      Yield, acreage, income.
   \item{NuGIS} \cite{nuGIS}
       Soil $N$, $P$ , $K$ addition and withdrawal.
   \item{CDC WONDER} \cite{modis,nldas}
      Air and surface temperatures, precipitation.
   \item{HWW RPN} \cite{hww-rpn}
      Testing of advanced breeding lines.
\end{itemize}


For all data, linear regression coefficients on year $x_i$ and county $n$ were computed in R using the model $y_{n i} = a_{n} + b_{n} x_i + e_{n i}$. Weighted means were computed by by $\bar{y}_{n} = \hat{a}_{n} + \hat{b}_{n} \times 1999$. Slopes were normalized to a relative rate as $\% \hat{b_n} = 100 \times \hat{b}_{n} / \bar{y}_{n}$.

}

%----------------------------------------------------------------------------------------
%	CONCLUSION
%----------------------------------------------------------------------------------------

\headerbox{Conclusion}{name=conclusion,column=0,below=methods}{

\smaller This analysis demonstrates that multiple data sources can be integrated to provide insight into regional yield trends. However, difficulties managing heterogenous data, too numerous to list here, present a challenge to updating this analysis as new data are available.

}

%----------------------------------------------------------------------------------------
%	REFERENCES
%----------------------------------------------------------------------------------------

\headerbox{References}{name=references,column=0,below=conclusion}{

\smaller % Reduce the font size in this block
\renewcommand{\section}[2]{\vskip 0.05em} % Get rid of the default "References" section title
%\nocite{*} % Insert publications even if they are not cited in the poster

\bibliographystyle{plain}
\bibliography{sample} % Use sample.bib as the bibliography file
}


%----------------------------------------------------------------------------------------
%	RESULTS 1
%----------------------------------------------------------------------------------------

\headerbox{Spatial Plots, Winter Wheat Yield and Covariates}{name=results1,span=2,column=1,row=0}{ % To reduce this block to 1 column width, remove 'span=2'

\begin{center}
\includegraphics[width=0.49\linewidth]{WheatYields-yielda}
\includegraphics[width=0.49\linewidth]{WheatYields-yieldb}
\captionof{figure}{a. Average winter wheat yields by county. b. Yearly percent change in county winter wheat yields}
\end{center}

\begin{center}
\includegraphics[width=0.24\linewidth]{WheatYields-maxb}
\includegraphics[width=0.24\linewidth]{WheatYields-heatb}
\includegraphics[width=0.24\linewidth]{WheatYields-precipb}
\includegraphics[width=0.24\linewidth]{WheatYields-sunb}
\end{center}
\begin{center}
\includegraphics[width=0.24\linewidth]{WheatYields-RatioP2O5}
\includegraphics[width=0.24\linewidth]{WheatYields-wheatfrac}
\includegraphics[width=0.24\linewidth]{fitFullLinearModels-rpn}
\includegraphics[width=0.24\linewidth]{WheatYields-moran}
\captionof{figure}{Spatial distribution of changing covariates correlated with yield changes. Alphabetic identifiers correspond to Table \ref{coeftable}. Figures A,B,C,D,H and L represent county level changes over from 1984-2014. Figure M shows change in the average yields for entries in the HWW RPN. Figure N plots a spatial autocorrelation measure (Moran's I), and shows two areas where yield changes cluster. M and N are not included in the regression shown in Table \ref{coeftable}.}
\end{center}
%------------------------------------------------


}



\headerbox{Regression Coefficients}{name=results3,span=1,column=1,below=results1,above=bottom}{ % To reduce this block to 1 column width, remove 

The linear model relating change in yield ($Y_b$) to relative change in covariates, $Y_b = \% \hat{b_1} + \% \hat{b_2} + \dots + \% \hat{b_n}$, was fit in R using the {\tt lm} function. An optimal linear model was found using the {\tt step} function. Table \ref{coeftable} shows the result of this function. Coefficients listed are in \% units, with the exception of Percent Harvested Acres, Wheat, which was computed as the percent wheat acres harvested of total crop acres harvested.


\begin{center}

\begin{tabular}{l r l}
\toprule
\textbf{Covariate} & \textbf{Coefficient} & \\
\midrule
A. Daily Max Air Temperature \cite{nldas} & -1.1844  &  *** \\ %max.b
B. Daily Max Heat Index \cite{nldas} &  6.1933  &  *** \\ %heat.b
C. Daily Precipitation  \cite{nldas} &  -0.1844  &  \\ %precip.b
D. Daily Sunlight   \cite{nldas}  & -2.7275  & ** \\ %sun.b
E. Day Land Surface Temperature \cite{modis} &  0.1157   &  *  \\ %day.b
F. Farm Inputs, Tons $P$ \cite{nuGIS} & -0.0110 \\ %Farm TonsP.b
G. Nutrient Removal by Crops, $N$ \cite{nuGIS} & -0.0119  &  \\   %Tons_N_Rem.b
H. Net Removal to Use Ratio, $P_2 O_5$  \cite{nuGIS} & -0.0385  &  * \\ %RatioP2O5.b 
I. Sum of 21 Crop Acres Planted  \cite{nuGIS} & -0.4795   &  * \\ 
J. Sum of 21 Crop Acres Harvested  \cite{nuGIS} &  0.7501  &  *** \\
K. Total Cropland \cite{nuGIS} & -0.1107  &  *  \\%TotalCropland.b
L. Percent Harvested Acres, Wheat               & -0.0079 &  *** \\ %wheat frac
 

\bottomrule
\end{tabular}
\captionof{table}{Summary of best linear model coefficients. Coefficients of regression are significant at 0.1\% (***) 1\% (**) 5\% (*) and 10\%(.)}
\label{coeftable}
\end{center}

%------------------------------------------------


%------------------------------------------------

%------------------------------------------------


}

\headerbox{Principal Components}{name=results2,span=1,column=2,below=results1,above=bottom}{ % To reduce this block to 1 column width, remove 'span=2'

\begin{center}
\includegraphics[width=0.85\linewidth]{WheatYields-pca2}
\captionof{figure}{Principle components of covariates associated with yield changes. See Table \ref{coeftable} for large letter codes. Small letter codes represent data points for counties, denoted by state.}
\end{center}

%------------------------------------------------
}

%----------------------------------------------------------------------------------------

\end{poster}

\end{document}